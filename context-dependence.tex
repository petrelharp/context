\documentclass{article}
\usepackage{fullpage}
\usepackage{amsmath,amssymb}
\usepackage{amsthm}

\newcommand{\E}{\mathbb{E}}
\renewcommand{\P}{\mathbb{P}}
\newcommand{\calS}{\mathcal{S}}  % set of states
\newcommand{\calT}{\mathcal{T}}  % set of transition triples
\newcommand{\nA}{\mbox{A}}  % nucleotides:
\newcommand{\nC}{\mbox{C}}
\newcommand{\nG}{\mbox{G}}
\newcommand{\nT}{\mbox{T}}
\newcommand{\join}{\oplus}  % matches
\newcommand{\st}{\colon}  % such that
\newcommand{\var}{\mathop{\mbox{var}}}
\newcommand{\cov}{\mathop{\mbox{cov}}}

\theoremstyle{plain}
\newtheorem{theorem}{Theorem}
\theoremstyle{definition}
\newtheorem{example}{Example}[section]

\begin{document}

\section*{Introduction}

Interacting particle systems with neighborhood structure: transition probabilities generally not expressable (except TASEP).
Markov random fields with Glauber dynamics a case of this.

Example: Ising model.

Context-dependent substitution processes are an example of interest for phylogenetics and evolutionary inference.
Usually dealt with either by using larger and larger windows (but, only up to four bp),
or by doing MCMC over additional information (entire history of changes).
Also, Markov-along-the-genome method.

The difficulty comes from the fact that the rates of change at any one site depend in principle on arbitrarily far away sites,
so transition matrix is ginormous.

Goal: parameter inference, given before-after observations, or else observations descended from a common ancestor.

Review stats lit, e.g. reviews by Friel, original by Besag (autologistic model), genetics by Larribe and Fearnead.

Here restrict to one-dimensional neighborhood structure,
to make it easier.

\section{Notation and Examples}

The common framework of the above examples is a one-dimensional grid of $n$ sites,
and each site is labeled with one of a finite set of states.
These change with time -- so, write $X_i(t)$ for the state of the $i^\mathrm{th}$ site at time $t$,
where each $X_i(t) \in \calS$, the set of possible states.
(To avoid edge effects, we sometimes arrange these sites in a circle, so the indices are interpreted $\mod n$.)
The time evolution is Markov, and is determined by a set of transition rates and associated patterns.  
``Patterns'' are sequences of states, and we let $|u|$ denote the length of the pattern $u$.
Transition rates are assumed to be local and homogeneous, so for example saying ``pattern $u$ changes to $v$ at rate $\mu$''
means that if $(X_i(t), \ldots, X_{i+\ell}(t)) = (u_1, \ldots, u_\ell)$ (where $\ell=|u|=|v|$),
then $\P\{(X_i(t+dt), \ldots, X_{i+\ell}(t+dt)) = (v_1, \ldots, v_\ell) = \mu dt + o(dt)$,
regardless of the location $i$ along the sequence.
We call such a $(\mu,u,v)$ a ``transition triple''.
The set of all transition triples $\calT = \{ (\mu_j,u_j,v_j) \}_{j=1}^{n_T}$ 
determines the time evolution of the process.

It will often be convenient, notationally and computationally, to avoid edge effects
by treating the $n$ sites as a circle, and treating all indices $\mod n$.

\begin{example}[TASEP]
  The \emph{Totally Asymmetric Simple Exclusion Process} labels each site as ``empty'' or ``occupied'' (0 or 1, respectively), 
  and says that each occupied site, independently at rate $\lambda$, checks the site to the right to see if it is empty,
  and if it is, moves there.  
  This process therefore has only one transition triple:

  \begin{center}
    \begin{tabular}{c@{\quad$\to$\quad}c@{\quad at rate\quad }c}
      $u$  &  $v$  &  $\mu$  \\
      \hline
      01  &   10   &  $\lambda$
    \end{tabular}
  \end{center}

  This only has one parameter, the speed.  
  Note, however, that it may not be entirely trivial to estimate the speed,
  since (on the circle) it may not be obvious which particle moved where.

\end{example}

\begin{example}[CpG mutation]
  A sequence of genome can be written using A, C, G, and T;
  in the most general model of single-site mutation, each of the 12 possible transitions occurs at its own rate.
  Furthermore, it is a well-known observation that in many species, and certain contexts, 
  a C followed by a G has an additional, higher rate with either the C changing to a T or the G changing to an A.
  This simplest model of context-dependent substitution rates in a genome sequence has rates

  \begin{center}
    \begin{tabular}{c@{\quad$\to$\quad}c@{\quad at rate\quad }cc}
      $u$  &  $v$  &  $\mu$  & \\
      \hline
      $x$  &  $y$  &  $m_{xy}$ & \qquad $x \neq y \in \{\nA,\nC,\nG,\nT\}$  \\
      CG   &  TG   &  $\gamma$ & \\
      CG   &  CA   &  $\gamma$ &  
    \end{tabular}
  \end{center}

  This has 13 parameters.
  Note that there are no conserved quantities here, unlike in TASEP,
  which we could imagine as indestructible particles moving.


\end{example}


\begin{example}[Ising model with Glauber dynamics]
  In the Ising model, each site is labeled as either ``up'' or ``down'' ($+1$ or $-1$ respectively),
  imagined as magnetic dipoles,
  and the energy associated with a given state $x$ is $H(x) = - \beta \sum_i x_i x_{i+1} - \gamma \sum_i x_i$.
  The associated stationary distribution on configurations is proportional to $\exp(-H(x))$.
  One way to add temporal dynamics that preserve the distribution 
  is to say that each site, independently at rate $\lambda$,
  forgets its spin, 
  and reconfigures to a state chosen with probability proportional to the stationary probability of the resulting configuration.
  This is known as ``Glauber dynamics'', and ignoring transitions that don't change the state,
  we get that if patterns $u$ and $v$ differ at only one site, then $u$ changes to $v$ at rate $1/(1+\exp(H(u)-H(v)))$, so

  \begin{center}
    \begin{tabular}{c@{\quad$\to$\quad}c@{\quad at rate\quad }c}
      $u$  &  $v$  &  $\mu$  \\
      \hline
      $+++$  &   $+-+$   &  $\lambda/1+e^{2\beta - \gamma})$ \\
      $++-$  &   $+--$   &  $\lambda/(1+e^{-\gamma})$ \\
      $+-+$  &   $+++$   &  $\lambda/(1+e^{-2\beta + \gamma})$ \\
      $+--$  &   $++-$   &  $\lambda/(1+e^{\gamma})$ \\
      $-++$  &   $--+$   &  $\lambda/(1+e^{-\gamma})$ \\
      $-+-$  &   $---$   &  $\lambda/(1+e^{-2\beta - \gamma})$ \\
      $--+$  &   $-++$   &  $\lambda/(1+e^{\gamma})$ \\
      $---$  &   $-+-$   &  $\lambda/(1+e^{2\beta + \gamma})$ 
    \end{tabular}
  \end{center}

  This has three parameters: the speed $\lambda$, the inverse temperature $\beta$, and the strength of magnetization $\gamma$
  (here scaled by temperature).

\end{example}


\paragraph{The generator matrix}
For clarity, we should define how the set of transition rates determines the transition rate matrix,
the $|\calS|^n \times |\calS|^n$ matrix $G(n)$ whose $(x,y)^\text{th}$ entry gives the instantaneous rate 
with which the process in state $x$ jumps to state $y$.
Let $x_i^{(\ell)} = (x_i, x_{i+1}, \ldots, x_{i+\ell-1})$ be the subsequence of length $\ell$ beginning at location $i$,
and for each $1\le i \le n$, and patterns $u,v \in \calS^\ell$ define the relation
\[
x \xrightarrow{i,u,v} y \qquad \text{iff} \qquad \begin{cases}
  x_j = y_j \quad &\text{for } j<i \\
  x_{i+k} = u_k \quad &\text{for } k < \ell \\
  y_{i+k} = v_k \quad &\text{for } k < \ell, and \\
  x_j = y_j \quad &\text{for } j\ge i+\ell ,
\end{cases}
\]
i.e.\ if $x$ and $y$ match except for in $i,i+1,\ldots,i+\ell$, where $x$ matches with $u$ and $y$ matches with $v$.
Interpet these indices $\mod n$, even for short sequences, a point we will come back to.
We'll also need the set of transitions that can move from $x$ to $y$ at a given location: let  $J(i,x,y) = \{ j \st x \xrightarrow{i,u_j,v_j} y \}$.
Then, the rate $G(n)_{x,y}$ is the sum of all matching transition rates,
namely
\begin{align} \label{eqn:G_defn}
  G(n)_{x,y} = \sum_{i=1}^n \sum_{j \in J(i,x,y)}  \mu_j ,
\end{align}
and if there are no triples $(\mu,u,v)$ with $x \xrightarrow{i,u,v} y$ for some $i$, then $G(n)_{x,y}=0$.
In principle, this gives us the transition probabilities for the process:
\begin{align} \label{eqn:full_likelihood}
  p_n(t;x,y) := \P\{ X(t) = y \mid X(0) = x \} = (e^{tG_n})_{xy} .
\end{align}


\section{Simulation}

First, it will be useful to have a brief description of how to simulate such a process.
Briefly, we can put down a Poisson process of times and locations of possible changes,
at an appropriate rate $\mu_*$,
and then resolve each possible change in temporal order.
We situate ``changes'' at the beginning of the matching pattern,
so $\mu_*$ can be defined as the maximum sum of rates matching  the start of a sequence:
\[
\mu_* = \max \left\{ \sum_{y \in \calS^n} \sum_{j\in J(1,x,y)} \mu_j \st x \in \calS^n \right\} .
\]
Now sample possible changes: let $s_{ij}$ denote the $j^\text{th}$ possible change at site $i$,
so that $0 \le s_{i1} < s_{i2} < \cdots < s_{iN_i} \le t$ is a rate $\mu_*$ Poisson process.
Then, suppose we have determined the state at time $s$ to be $X(s) = x$,
and the next event is $s_{ij} = \min \{ s_{k\ell} : s_{k\ell}>s \}$.
The process then jumps to a new state at times $s_{ij}$ with probability proportional to the corresponding rate.
In other words, let
\[
q_k = \begin{cases}
  \mu_k/\mu_* \qquad & \text{if } x_i^{(|u_k|)} = u_k  \\
  0 \qquad & \text{otherwise,}
\end{cases}
\]
and let $q_0 = 1-\sum_{k=1}^{n_T} q_k$.
By construction of $\mu_*$, $q_0\ge 0$.)
Now choose transition $k$ with probability $q_k$, and if $k>0$, replace $x_i^{|u_k|}$ with $v_k$.

For instance, to simulate TASEP, one has only to let $\mu_*=\lambda$, and at each $s_{ij}$
check if at site $i$ there is a 1 followed by a 0,
and if so, switch them.


\section{Inference}

The problem at hand is to infer the parameters of the model, given the state at time 0 and later at time $t$.
If $n$ is large enough and $t$ is not too large,
this should be feasible.
(If $t$ is too large, the process may ``saturate'' --
if there are many changes at most sites, then in at least most models, 
we will lost most of the information about the dynamics,
retaining information only about the parameters that affect the stationary distribution.
In TASEP, we lose all information as it approaches stationarity,
in the Ising model we have information about $\beta$ and $\gamma$ but not $\lambda$,
and in the CpG model it is not immediately clear.


But, how can we extract the information?
These are Markov processes, on the state space $\calS^n$, 
so the full likelihood function is given by \eqref{eqn:full_likelihood},
but doing anything with the $|\calS|^n \times |\calS|^n$ matrix $G_n$ is clearly infeasible.
We can, however, compute \eqref{eqn:full_likelihood} for smaller $n$,
so the first thing that one might think to do is to break the sequence up into many blocks of length $m$,
and treat these as independent, taking $m$ as large as is feasible.
Then we would define $p_m(t;x,y)$ to be the probability that a string $x$ of length $m$ evolves to string $y$ over time $t$,
denote by $x_i^{(m)}$ the block of length $m$ beginning at index $i$,
and obtain the approximate likelihood function
\[
  \prod_{k=0}^{n/m} p_m(x_{km}^{(m)},y_{km}^{(m)}) .
\]
Variations are possible (what do we do about the edge effects for $p_m$?),
but in any case there is no guarentee that this is close to correct,
as it ignore dependencies between neighboring blocks.

Intuitively, we want to increase the size of the context 
-- but increasing $m$ does not do this, as there are always edge effects.
However, we could look at the context in the initial sequence but not in the final sequence,
including the states of an additional $\ell$ and $r$ sites on the left and right, respectively.
Concretely, define the marginal probabilities
\begin{align}
  p_{\ell,m,r}(t;x,y) &= \sum_{a \in \calS^\ell} \sum_{b \in \calS^r} p_{\ell+m+r}(t;x,a \join y \join b) , \quad \text{for}\; x \in \calS^{\ell+m+r}, \; y \in \calS^m ,
\end{align}
where $a \join y \join b$ is the sequence composed of concatenating $a$, $y$, and $b$ together in that order.
This is the probability that if we begin with a sequence $x$ of length $\ell+m+r$, 
then after time $t$ we will see at the middle $m$ positions the pattern $y$,
ignoring what is present in the leftmost $\ell$ and rightmost $r$ positions.
In a long sequence, $p_m(x,z)$ might not be a very good approximation for $\P\{ X_i^{(m)}(t) = z | X_i(0)^{(m)} = x \}$,
we do expect $p_{\ell,m,r}(x,y)$ to be correct, asymptotically in $\ell$ and $r$:
\begin{align} \label{eqn:window_approx}
  \P\{ X_i^{(m)}(t) = y | X_i(0)^{(\ell+m+r)} = x \} \approx p_{\ell,m,r}(x,y) .
\end{align}
We provide a concrete statement of this approximation later.

%%%%
\subsection{A likelihood function}

Now we know how to compute the (marginal) probability that a particular subsequence changes into another subsequence.
For patterns $x$ and $y$, with offset $\ell$ (so that $|x|=|y|+2\ell$),
we can count how many times $x$ appears in $X(0)$,
and how many times $x$ appears in $X(0)$ while $y$ appears at the same place in $X(t)$, offset by $\ell$.
For instance, if $x$ is ACTCAC, $y$ is CGC, and $\ell=3$, this looks like:
\begin{center}
\begin{tabular}{c|ccccccc}
 $x$ &  C  & A & C & T & C & A & C \\
 $y$ &  $\centerdot$  & $\centerdot$  & C & G & C & $\centerdot$  & $\centerdot$  
\end{tabular}
\end{center}
where $\cdot$ could be any base.
Let $N(x)$ and $N_\ell(x,y)$ count these events, respectively.

From the definitions above, 
the expected number of times $y$ would appear juxtaposed with $x$ is approximately $N(x)p_{\ell,m,r}(x,y)$.
Covariances are also easy to compute.
If we define $M_i(x,y)$ to be 1 if $X_{i-\ell}^{|x|}(0)=x$ and $X_i^({|y|})(t)=y$, and 0 otherwise,
so that $N(x,y) = \sum_{i=1}^n M_i(x,y)$, then
\begin{align}
  \var[N(x,y)] &= \sum_{i=1}^n \var[M_i(x,y)] + \sum_{i\neq j} \cov[M_i(x,y), M_j(x,y)] \\
  &= n \var[M_1(x,y)] + 2n\sum_{k=1}^n \cov[M_1(x,y),M_{k+1}(x,y)] \\
  &\approx n p_{\ell,m}(x,y)(1-p_{\ell,m}(x,y)) + 2n\sum_{k=1}^{m+2\ell} \left( \E[M_1(x,y) M_{k+1}(x,y)] - p_{\ell,m}(x,y)^2 \right) \\
  &= n p_{\ell,m}(x,y)(1-(1+2m+4\ell)p_{\ell,m}(x,y)) + 2n\sum_{k=1}^{m+2\ell} \E[M_1(x,y) M_{k+1}(x,y)] .
\end{align}
The term $\E[M_1(x,y) M_{k+1}(x,y)]$ is the probability that $x$ matches at both site 1 and site $k+1$,
and that $y$ matches both at site $\ell+1$ and at site $\ell+k+1$.
For this to be nonzero, the joint pattern $(x,y)$ must allow self-overlaps.
Similarly, if $(x_1,y_1) \neq (x_2,y_2)$ then
\begin{align}
  \begin{split}
    \cov[N(x_1,y_1),N(x_2,y_2)] 
    & \approx - n (2m+4\ell) p_{\ell,m}(x_1,y_1)p_{\ell,m}(x_2,y_2) \\
    & \qquad {} + n\sum_{k=1}^{m+2\ell} \E[M_1(x_1,y_1) M_{k+1}(x_2,y_2) + M_1(x_2,y_2) M_{k+1}(x_1,y_1)] .
  \end{split}
\end{align}




%%%%
\subsection*{Proof of the approximation}

\begin{proof}

In fact, the approximation improves exponentially as $\ell$ and $m$ increase --
here is a simple argument.
Suppose the maximum length of any transition pattern is $R = \max\{ |u| : (\mu,u,v) \in \calT \}$.
The transition rate at site $i$ at time $s$ is determined if we know $X_{i-R}^{(2R+1)}(s)$;
this can be determined from $X_{i-R}^{2R+1}(0)$ and the outcome of all changes happening before time $s$ at sites in $(i-R-1, \ldots, i+R)$.
These will depend in turn on the outcome of previous possible changes.
For each site $i$ and times $s_1 < s_2$, define the set of states $H(i,s_1,s_2)$ 
so that if one knows the possible change points $s_{ij}$,
then changes outside of $H(i,s_1,s_2)$ do not affect the distribution of $X_i(s_2)$.
Concretely, define $H(i,s_1,s_2)$ by the property that $i \in H(i,s_1,s_2)$, and
\begin{gather}
  \text{if } s_1 < s_{jk} \le s_2 \text{ for some } i-R-1 \le j \le i+R, \text{ then } H(j,s_1,s_{jk}) \subseteq H(i,s_1,s_2) .
\end{gather}
In other words, $H(i,s_1,s_2)$ is the union of the $R$-neighborhoods of $i$ and all possible changes occurring in $H(i,s_1,s_2)$ between $s_1$ and $s_2$.
For a set of states $I$, let $H(I,s_1,s_2) = \bigcup_{i\in I} H(i,s_1,s_2)$.
Formally, the state $X_i(s_2)$ is conditionally independent of $\{X_j(s_1) \st j \notin H(i,s_1,s_2)\}$.
Informally, the approximation \eqref{eqn:window_approx} will be good if $H$ does note expand beyond the flanking context.
Now, MAKE THIS EASIER TO READ
\begin{multline}
  \P\{ X_i^{(m)}(t) = y \mid X(0) \} 
  = \P\left\{ H(\{i,\ldots,i+m\},0,t)\subseteq\{i-\ell,\ldots,i+m+r\} \right\} \\
   \qquad \qquad {} \times \P\left\{ X_i^{(m)}(t) = y \mid X_{i-\ell}^{(\ell+m+r)}(0), \; H(\{i,\ldots,i+m\},0,t)\subseteq\{i-\ell,\ldots,i+m+r\} \right\} \\
     \qquad {} + \P\left\{ H(\{i,\ldots,i+m\},0,t)\nsubseteq\{i-\ell,\ldots,i+m+r\} \right\}  \\
   \qquad\qquad {} \times \P\left\{ X_i^{(m)}(t) = y \mid X_{i-\ell}^{(\ell+m+r)}(0), \; H(\{i,\ldots,i+m\},0,t)\nsubseteq\{i-\ell,\ldots,i+m+r\} \right\} ,
\end{multline}
and so
\begin{align} \label{eqn:prob_approx}
  \left| \P\{ X_i^{(m)}(t) = y \mid X(0) \} - p_{\ell,m,r}(t;x,y) \right| \le 2 \P\left\{  H(\{I,\ldots,i+m\},0,t)\subseteq\{i-\ell,\ldots,i+m+r\} \right\}.
\end{align}

So, the error is bounded by the probability that $H(I,0,t)$ is larger than the context from $I-\ell$ to $I+r$.
However, following $H$ backwards in time, the outer edge of $H$ expands as a continuous-time random walk:
for instance, the left edge $L(s) = \min \{ j \in H(I,t-s,t) \}$ jumps leftwards at rate $R\mu_*$ with a random jump size
that has mean $R/2$.
General facts about random walks imply that \eqref{eqn:prob_approx} is bounded above by $C e^{-\alpha \ell/t}$, for some constants $C$ and $\alpha$.

\end{proof}

%%%%%%%%
\subsection{Edge effects and cyclicization}

The choice to treat long sequences as cyclical is generally of little importance.
However, we have defined in \eqref{eqn:G_defn} the basic building block of our approximation, $G(m)$,
to be the transition matrix \emph{for a circular sequence} of length $m$, where $m$ may be short.
This seems strange -- but, what else would we do at the edges?
We want to use $G(\ell+m+r)$ to compute the chance that a given $(\ell+m+r)$-sequence ends up with another given $m$-sequence in the middle;
and if $\ell$ and $r$ are big enough relative to $t$, this will be a good approximation regardless of what we do at the edges.
One answer would be to not allow patterns hanging off of either end to match,
resulting in a slight underestimate of the chance of change.
A better approximation would be to use the marginal frequencies in some way to average over surroudning sequence 
-- but, this requires marginal frequencies to be known, and these are not stationary.
However, the leftmost site in a randomly chosen window is a sample from the marginal distribution,
and if sites separated by distance $\ell+m+r$ are sufficiently uncorrelated,
wrapping the sequence around is a good approximation to averaging over marginal frequencies in the surrounding sequence.
It is easy to imagine situations where this would not be the best solution --
but again, if $\ell$ and $r$ are large enough, it won't matter much.




%%%%%%%%%%%
\section{Computation}






\end{document}
