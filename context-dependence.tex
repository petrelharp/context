\documentclass{article}
\usepackage{amsmath,amssymb}
\newcommand{\E}{\mathbb{E}}
\renewcommand{\P}{\mathbb{P}}

\begin{document}

\section*{Introduction}

Interacting particle systems with neighborhood structure: transition probabilities generally not expressable (except TASEP).
Markov random fields with Glauber dynamics a case of this.

Example: Ising model.

Context-dependent substitution processes are an example of interest for phylogenetics and evolutionary inference.
Usually dealt with either by using larger and larger windows (but, only up to four bp),
or by doing MCMC over additional information (entire history of changes).
Also, Markov-along-the-genome method.

The difficulty comes from the fact that the rates of change at any one site depend in principle on arbitrarily far away sites,
so transition matrix is ginormous.

Goal: parameter inference, given before-after observations, or else observations descended from a common ancestor.

Here restrict to one-dimensional neighborhood structure,
to make it easier.

\section*{Notation}




\end{document}
