%%%%%%
%%
%%  Don't reorder the reviewer points; that'll mess up the automatic referencing!
%%
%%%%%

\begin{minipage}[b]{2.5in}
  Resubmission Cover Letter \\
  {\it Evolution}
\end{minipage}
\hfill
\begin{minipage}[b]{2.5in}
    Joshua Schiffman \\
    \emph{and} Peter Ralph \\
  \today
\end{minipage}
 
\vskip 2em
 
\noindent
{\bf To the Editor(s) -- }
 
\vskip 1em

Our revised manuscript.

\noindent \hspace{4em}
\begin{minipage}{3in}
\noindent
{\bf Sincerely,}

\vskip 2em

{\bf 
Erick Matsen and
Peter Ralph
}\\
\end{minipage}

\vskip 4em

\pagebreak

%%%%%%%%%%%%%%
\reviewersection{1}


\begin{quote}
In their article ``Enabling inference for context-dependent models of
mutation by bounding the propagation of dependency'' Matsen and Ralph
discuss methodology to infer mutation rates in context dependent
models of DNA evolution. With the advent of more sequence data it
became clear that the first very simple models of DNA evolution are
only a crude approximation to the ``real'' process. The extension of
such models to also include neighbor dependencies is very timely.
However, the inclusion of such processes is easier said than done -
without ``bounding the propagation of dependency'' such models are
computational not tractable.

The paper is very well written and gives sufficient detail.

The source code of the mentioned R package is available at github.com.
\end{quote}

Thanks very much for the kind words!

%%%%%%%%%%%%%%
\reviewersection{2}


\begin{quote}
In this manuscript the authors address the problem how to infer model
parameters and model ``ingredients'' (allowed transitions) in models of
sequence evolution, in which the rate of mutations depends on the
local sequence context of a site. This is a significant problem and
surprisingly difficult. The authors present a solution where they
decompose the full transition matrix into elements that map
subsequences into even shorter subsequences. This is an approximation,
but they give a bound on the approximation and even more importantly
show that the method works on two sets of simulated data. They then
apply the method to real sequence data from humans and chimpanzee and
find interesting rates for different processes relevant for sequence
evolution.
\end{quote}

% > This is overall a very nice work, which in addition is presented very
% > lucidly, but there are a few points the authors should address before
% > publication:

\begin{point}{}
Could the authors provide some thoughts about sequences with
periodic boundary conditions. That would be a natural setup both for
the TASEP that they use as one of their examples and for bacterial
genomes. Would they expect the method to still work and/or does
anything need to be changed?
\end{point}

\reply{
This is something that we thought about a good deal early on in this project,
but removed to avoid introducing additional complexities.
We agree it's a natural question to ask,
and so have inserted some dicussion \revref.
}

\begin{point}{}
Since the authors in the introduction bring up contexts like wild
fires, land use patterns, and cell-cell interactions, could the
authors in the discussion at least speculate about the possibility of
extending this approach to 2D or 3D systems?
\end{point}

\reply{
Good idea; we've inserted a brief note. \revref
}

\begin{quote}
3) There are a few places where the presentation could be improved:
\end{quote}

\begin{point}{}
It is somewhat confusing that in the definition of the
% x --i,u,v--> y
$ x \xrightarrow{i,u,v} y$
relation toward the end of section 1 \revref{}
the indices $i$ of the sequences
$x$ and $y$ are one-based while the indices $k$ of the patterns $u$ and $v$ are
zero-based. Can this be made more consistent?
\end{point}

\reply{
Thanks for pointing this out,
but after trying a few solutions out
we'd like to keep it as-is.
We think that the indexing on transition triples (for $u$ and $v$)
should certainly be zero-based, because they are offsets.
On the other hand, there would be no problem doing zero-based indexing for sequences ($x$ and $y$),
but since these indexes are absolute positions it seems less important,
and we think that $\sum_{i=1}^n$ is slightly more clear than $\sum_{i=0}^{n-1}$.
(We tried inserting a note about this, but felt that even the note made things more confusing.)
}

\begin{point}{}
It is quite confusing that the length of the sequence is $n$ in
section 1 and then switches to $L$ in section 2. Later $n$ becomes the
length of a subsequence block. The notation should ideally be made
consistent between the sections, or, at a minimum, the reader should
be warned of these notation changes.
\end{point}

\reply{
This is a good point; we've made this change.
(We had been trying to arrange things to introduce the generator matrix
using the notation $G(n)$ rather than $G(L)$,
as that's how it would mostly appear later,
but this was not making this more clear.)
}

\begin{point}{}
This referee would take some objection to calling the simulation
algorithm in section 2 a Gillespie algorithm. The whole point of the
Gillespie algorithm is to advance time in such a way that every step
of the algorithm results in a transition. This is not the case in the
algorithm the authors present since they always step forward with the
maximal possible rate but then allow non-transitions.
\end{point}

\reply{
We don't entirely agree here:
one could also say that
the point of the Gillespie algorithm is to obtain an exact trajectory
of a continuous-time system
by only considering those discrete times when there is a transition.
This view doesn't disallow ``do-nothing'' transitions.
We wouldn't mind removing the phrase entirely,
but on the other hand the phrase is widely used
(see for instance a nice Wikipedia page and 8,500 google scholar hits),
so we'd like to leave it in to give readers new to the subject
some more pointers into the background literature.
}



\begin{point}{3d}
There is an inconsistency between equations (5), (6), and the
equation right below (6). In the numerator the identification between
the symbols must be that (n,l,r) in (5) is (2l+1,l+1,l) in (6).
However, the upper index on x is n in (5) but 4l+1 != 2l+1 in (6).
That is true in the denominator as well. Also,
m=n-l-r=(2l+1)-(l+1)-l=0 != 2l+1 (if the authors meant that n was
actually 4l+1, 4l+1-(l+1)-l=2l != 2l+1 does not work much better
either) and in the denominator it is not clear why m should change
given that l and r just get swapped and m only depends on their sum.
Maybe this referee totally misunderstands the notation here, but this
fact alone means that something needs to be improved in the
presentation in this area.
\end{point}

\reply{
TODO: Sorry about this. I suggest the solution is to change the
first subscript on p to 4l+1.
}

\begin{point}{3e}
It is somewhat counterintuitive that the caption of figure 4 talks
of ``bases''. This seems like taking the analogy a bit too far.
Shouldn't it be ``spins''?
\end{point}

\reply{
Good point; done.
}

\begin{point}{3f}
Remove a ``that'' in ``that that'' in the fifth line of appendix C.
\end{point}

\reply{
Fixed, thanks.
}
